\documentclass[12pt]{article}
%%---------------------------------------------------------------------
% packages
% geometry
\usepackage{geometry}
% font
\usepackage{fontspec}
\defaultfontfeatures{Mapping=tex-text}  %%如果没有它,会有一些 tex 特殊字符无法正常使用,比如连字符。
\usepackage{xunicode,xltxtra}
\usepackage[BoldFont,SlantFont,CJKnumber,CJKchecksingle]{xeCJK}  % \CJKnumber{12345}: 一万二千三百四十五
\usepackage{CJKfntef}  %%实现对汉字加点、下划线等。
\usepackage{pifont}  % \ding{}
% math
\usepackage{amsmath,amsfonts,amssymb}
% color
\usepackage{color}
\usepackage{xcolor}
\definecolor{EYE}{RGB}{199,237,204}
\definecolor{FLY}{RGB}{128,0,128}
\definecolor{ZHY}{RGB}{139,0,255}
% graphics
\usepackage[americaninductors,europeanresistors]{circuitikz}
\usepackage{tikz}
\usetikzlibrary{positioning,arrows,shadows,shapes,calc,mindmap,trees,backgrounds}  % placements=positioning
\usepackage{graphicx}  % \includegraphics[]{}
\usepackage{subfigure}  %%图形或表格并排排列
% table
\usepackage{colortbl,dcolumn}  %% 彩色表格
\usepackage{multirow}
\usepackage{multicol}
\usepackage{booktabs}
% code
\usepackage{fancyvrb}
\usepackage{listings}
% title
\usepackage{titlesec}
% head/foot
\usepackage{fancyhdr}
% ref
\usepackage{hyperref}
% pagecolor
\usepackage[pagecolor={EYE}]{pagecolor}
% tightly-packed lists
\usepackage{mdwlist}
\usepackage{verbatim}%comment命令的注释包
\usepackage{styles/iplouccfg}
\usepackage{styles/zhfontcfg}
\usepackage{styles/iplouclistings}

%%---------------------------------------------------------------------
% settings
% geometry
\geometry{left=2cm,right=1cm,top=2cm,bottom=2cm}  %设置 上、左、下、右 页边距
\linespread{1.5} %行间距
% font
\setCJKmainfont{Adobe Kaiti Std}
%\setmainfont[BoldFont=Adobe Garamond Pro Bold]{Apple Garamond}  % 英文字体
%\setmainfont[BoldFont=Adobe Garamond Pro Bold,SmallCapsFont=Apple Garamond,SmallCapsFeatures={Scale=0.7}]{Apple Garamond}  %%苹果字体没有SmallCaps
\setCJKmonofont{Adobe Fangsong Std}
% graphics
\graphicspath{{figures/}}
\tikzset{
    % Define standard arrow tip
    >=stealth',
    % Define style for boxes
    punkt/.style={
           rectangle,
           rounded corners,
           draw=black, very thick,
           text width=6.5em,
           minimum height=2em,
           text centered},
    % Define arrow style
    pil/.style={
           ->,
           thick,
           shorten <=2pt,
           shorten >=2pt,},
    % Define style for FlyZhyBall
    FlyZhyBall/.style={
      circle,
      minimum size=6mm,
      inner sep=0.5pt,
      ball color=red!50!blue,
      text=white,},
    % Define style for FlyZhyRectangle
    FlyZhyRectangle/.style={
      rectangle,
      rounded corners,
      minimum size=6mm,
      ball color=red!50!blue,
      text=white,},
    % Define style for zhyfly
    zhyfly/.style={
      rectangle,
      rounded corners,
      minimum size=6mm,
      ball color=red!25!blue,
      text=white,},
    % Define style for new rectangle
    nrectangle/.style={
      rectangle,
      draw=#1!50,
      fill=#1!20,
      minimum size=5mm,
      inner sep=0.1pt,}
}
\ctikzset{
  bipoles/length=.8cm
}
% code
\lstnewenvironment{VHDLcode}[1][]{%
  \lstset{
    basicstyle=\footnotesize\ttfamily\color{black},%
    columns=flexible,%
    framexleftmargin=.7mm,frame=shadowbox,%
    rulesepcolor=\color{blue},%
%    frame=single,%
    backgroundcolor=\color{yellow!20},%
    xleftmargin=1.2\fboxsep,%
    xrightmargin=.7\fboxsep,%
    numbers=left,numberstyle=\tiny\color{blue},%
    numberblanklines=false,numbersep=7pt,%
    language=VHDL%
    }\lstset{#1}}{}
\lstnewenvironment{VHDLmiddle}[1][]{%
  \lstset{
    basicstyle=\scriptsize\ttfamily\color{black},%
    columns=flexible,%
    framexleftmargin=.7mm,frame=shadowbox,%
    rulesepcolor=\color{blue},%
%    frame=single,%
    backgroundcolor=\color{yellow!20},%
    xleftmargin=1.2\fboxsep,%
    xrightmargin=.7\fboxsep,%
    numbers=left,numberstyle=\tiny\color{blue},%
    numberblanklines=false,numbersep=7pt,%
    language=VHDL%
    }\lstset{#1}}{}
\lstnewenvironment{VHDLsmall}[1][]{%
  \lstset{
    basicstyle=\tiny\ttfamily\color{black},%
    columns=flexible,%
    framexleftmargin=.7mm,frame=shadowbox,%
    rulesepcolor=\color{blue},%
%    frame=single,%
    backgroundcolor=\color{yellow!20},%
    xleftmargin=1.2\fboxsep,%
    xrightmargin=.7\fboxsep,%
    numbers=left,numberstyle=\tiny\color{blue},%
    numberblanklines=false,numbersep=7pt,%
    language=VHDL%
    }\lstset{#1}}{}
% pdf
\hypersetup{pdfauthor={Haiyong Zheng},%
            pdftitle={Title},%
            CJKbookmarks=true,%
            bookmarksnumbered=true,%
            bookmarksopen=false,%
            plainpages=false,%
            colorlinks=true,%
            citecolor=green,%
            filecolor=magenta,%
            linkcolor=cyan,%red(default)
            urlcolor=cyan}
% section
%http://tex.stackexchange.com/questions/34288/how-to-place-a-shaded-box-around-a-section-label-and-name
\newcommand\titlebar{%
\tikz[baseline,trim left=3.1cm,trim right=3cm] {
    \fill [cyan!25] (2.5cm,-1ex) rectangle (\textwidth+3.1cm,2.5ex);
    \node [
        fill=cyan!60!white,
        anchor= base east,
        rounded rectangle,
        minimum height=3.5ex] at (3cm,0) {
        \textbf{\thesection.}
    };
}%
}
\titleformat{\section}{\Large\bf\color{blue}}{\titlebar}{0.1cm}{}
% head/foot
\setlength{\headheight}{15pt}
\pagestyle{fancy}
\fancyhf{}
%\lhead{\color{black!50!green}2014年秋季学期}
\chead{\color{black!50!green}Ubuntu 14.04下安装TexLive2014并用XeTex配置中文字体}
%\rhead{\color{black!50!green}通信电子电路}
\lfoot{\color{blue!50!green}常琳(整理)}
%\cfoot{\color{blue!50!green}\href{http://vision.ouc.edu.cn/~zhenghaiyong}{CVBIOUC}}
\rfoot{\color{blue!50!green}$\cdot$\ \thepage\ $\cdot$}
\renewcommand{\headrulewidth}{0.4pt}
\renewcommand{\footrulewidth}{0.4pt}

%%---------------------------------------------------------------------
\begin{document}
%%---------------------------------------------------------------------
%%---------------------------------------------------------------------
% \titlepage
\title{\vspace{-2em}Ubuntu 14.04下安装TexLive2014并用XeTex配置中文字体\vspace{-0.7em}}
\author{常琳(整理)}
\date{\vspace{-0.7em}2015年8月\vspace{-0.7em}}
%%---------------------------------------------------------------------
\maketitle\thispagestyle{fancy}
%%---------------------------------------------------------------------
\maketitle
\tableofcontents 
%----------------------------------------------------------------------

\section{安装TexLive2014}

\begin{enumerate}
\item 下载TexLive2014镜像文件到/home/changlin
\item 挂载镜像文件:

在终端输入:sudo mount -o loop /home/changlin/software/texlive2014.iso /mnt
\item 进行安装

在终端输入:cd /mnt

           sudo ./install-tl

注意安装后默认的安装目录是/usr/local,要把安装文件移动到/opt下:

mv /usr/local/texlive /opt

这一步不是很快,要稍等一会儿

\begin{comment}
\item 配置环境变量:
  
  \begin{enumerate}
   \item 打开终端,输入:sudo vim ~/.profile  

         在最后添加以下代码:

        PATH=/opt/texlive/2014/bin/x86\_64-linux:\$PATH; export PATH
        MANPATH=/opt/texlive/2014/texmf-dist/doc/man:\$MANPATH; export MANPATH

        INFOPATH=/opt/texlive/2014/texmf-dist/doc/info:\$INFOPATH; export INFOPATH

   \item 然后sudo vim /etc/manpath.config
   在\# set up PATH to MANPATH mapping下输入

        MANPATH\_MAP /opt/texlive/2014/bin/x86\_64-linux (空格)/opt/texlive/2013/texmf-dist/doc/man
  
  \end{enumerate}

\end{comment}
\item 测试

此时texlive就安装成功了,你可以在/home下新建一个1.tex的文档,比如

$\backslash$documentclass\{article\}

$\backslash$begin\{document\}

dffdrgdfgfdgfd

$\backslash$end\{document\}

然后打开终端输入cd /home

               xelatex 1.tex

如果能生成一个PDF文件,里面的内容是dffdrgdfgfdgfd ,就说明安装成功了。

\end{enumerate}

\section{XeTex中文配置}

\begin{enumerate}

\item 在所新建测试文档的目录中新建一个zhfontcfg.sty的文件,然后使用命令sudo mktexlsr更新一下,文件的内容是

\% xetex/xelatex 字体设定宏包

$\backslash$ProvidesPackage\{zhfontcfg\}

$\backslash$usepackage\{fontspec,xunicode\}

$\backslash$defaultfontfeatures\{Mapping=tex-text\} \%如果没有它,会有一些 tex 特殊字符无法正常使用,比如连字符。

\% 中文断行

$\backslash$XeTeXlinebreaklocale "zh"

$\backslash$XeTeXlinebreakskip = 0pt plus 1pt minus 0.1pt

$\backslash$setCJKmainfont[BoldFont=\{SimHei\}, ItalicFont=\{楷体\_GB2312\}]\{SimSun\}

$\backslash$setCJKfamilyfont\{song\}\{SimSun\}

$\backslash$setCJKfamilyfont\{hei\}\{SimHei\}

$\backslash$setCJKfamilyfont\{kai\}\{楷体\_GB2312\}
 

\%楷体

$\backslash$newcommand\{$\backslash$kai\}\{$\backslash$CJKfamily\{kai\}\} 

$\backslash$def$\backslash$kaishu\{$\backslash$kai\}

\%黑体

$\backslash$newcommand\{$\backslash$hei\}\{$\backslash$CJKfamily\{hei\}\} 

$\backslash$def$\backslash$heiti\{$\backslash$hei\}

\%宋体

$\backslash$newcommand\{$\backslash$song\}\{$\backslash$CJKfamily\{song\}\} 

$\backslash$def$\backslash$songti\{$\backslash$song\}

\item 察看电脑中所安装过的字体:在终端输入:fc-list :lang=zh-cn,将安装包里的Adobe-fonts-otf文件夹复制到/usr/share/fonts/truetype下

sudo cp -a Adobe-fonts-otf /usr/share/fonts/truetype

\item 测试

新建一个名为2.tex的测试文档

文档内容

$\backslash$documentclass\{article\}

$\backslash$usepackage\{zhfontcfg\}

$\backslash$begin\{document\}

中文

$\backslash$end\{document\}

输入命令xelatex 2.tex,若生成一个内容为“中文”的PDF文档,则说明配置成功了, 如果不行,可以重启下试试。(注意,你的.tex文档应该与zhfontcfg.sty文件同在一个文件夹)。

\end{enumerate}

\end{document}
