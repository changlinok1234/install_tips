\documentclass[a4paper,12pt]{article}
%%--------------------------------------------------------------------
%packages
\usepackage{zhfontcfg}
\defaultfontfeatures{Mapping=tex-text}  %%如果没有它,会有一些 tex 特殊字符无法正常使用,比如连字符。
\linespread{1.5} %行间距
\title{Ubuntu 14.04下安装TexLive2014并用XeTex配置中文字体}
\author{常琳(整理)}
\date{2015-8-21}
\begin{document}
\maketitle
\section{安装TexLive2014}

\begin{enumerate}
\item 下载TexLive2014镜像文件到/home/changlin
\item 挂载镜像文件:

在终端输入:sudo mount -o loop /home/changlin/software/texlive2014.iso /mnt
\item 进行安装

在终端输入:cd /mnt

           sudo ./install-tl

注意安装后默认的安装目录是/usr/local,要把安装文件移动到/opt下:mv /usr/local/texlive /opt

这一步不是很快,要稍等一会儿

\item 配置环境变量:
  
  \begin{enumerate}
   \item 打开终端,输入:sudo vim ~/.profile  

         在最后添加以下代码:

        PATH=/opt/texlive/2014/bin/x86\_64(32为这里改为 i386,下同)-linux:\$PATH; export PATH
        MANPATH=/opt/texlive/2014/texmf-dist/doc/man:\$MANPATH; export MANPATH

        INFOPATH=/opt/texlive/2014/texmf-dist/doc/info:\$INFOPATH; export INFOPATH

   \item 然后sudo vim /etc/manpath.config
   在\# set up PATH to MANPATH mapping下输入

        MANPATH\_MAP /opt/texlive/2014/bin/x86\_64-linux (空格)/opt/texlive/2013/texmf-dist/doc/man
  
  \end{enumerate}
\item 测试

此时texlive就安装成功了,你可以在/home下新建一个1.tex的文档,比如

$\backslash$documentclass\{article\}

$\backslash$begin\{document\}

dffdrgdfgfdgfd

$\backslash$end\{document\}

然后打开终端输入cd /home

               xelatex 1.tex

如果能生成一个PDF文件,里面的内容是dffdrgdfgfdgfd ,就说明安装成功了。

\end{enumerate}

\section{XeTex中文配置}

我们的宏包文件中有zhfontcfg.sty,步骤1可省

\begin{enumerate}

\item 在所新建测试文档的目录中新建一个zhfontcfg.sty的文件,然后使用命令sudo mktexlsr更新一下,文件的内容是

% xetex/xelatex 字体设定宏包

$\backslash$ProvidesPackage\{zhfontcfg\}

$\backslash$usepackage\{fontspec,xunicode\}

$\backslash$defaultfontfeatures\{Mapping=tex-text\} %如果没有它,会有一些 tex 特殊字符无法正常使用,比如连字符。

% 中文断行

$\backslash$XeTeXlinebreaklocale "zh"

$\backslash$XeTeXlinebreakskip = 0pt plus 1pt minus 0.1pt

$\backslash$setCJKmainfont[BoldFont=\{SimHei\}, ItalicFont=\{楷体\_GB2312\}]\{SimSun\}

$\backslash$setCJKfamilyfont\{song\}\{SimSun\}

$\backslash$setCJKfamilyfont\{hei\}\{SimHei\}

$\backslash$setCJKfamilyfont\{kai\}\{楷体\_GB2312\}
 

%楷体

$\backslash$newcommand\{$\backslash$kai\}\{$\backslash$CJKfamily\{kai\}\} 

$\backslash$def$\backslash$kaishu\{$\backslash$kai\}

%黑体
$\backslash$newcommand\{$\backslash$hei\}\{$\backslash$CJKfamily\{hei\}\} 

$\backslash$def$\backslash$heiti\{$\backslash$hei\}

%宋体

$\backslash$newcommand\{$\backslash$song\}\{$\backslash$CJKfamily\{song\}\} 

$\backslash$def$\backslash$songti\{$\backslash$song\}

\item 察看电脑中所安装过的字体:在终端输入:fc-list :lang=zh-cn,将安装包里的Adobe-fonts-otf文件夹和wqy文件夹复制到/usr/share/fonts/truetype下

sudo cp -a Adobe-fonts-otf /usr/share/fonts/truetype

\item 测试

新建一个名为2.tex的测试文档

文档内容

$\backslash$documentclass\{article\}

$\backslash$usepackage\{zhfontcfg\}

$\backslash$begin\{document\}

中文

$\backslash$end\{document\}

输入命令xelatex 2.tex,若生成一个内容为“中文”的PDF文档,则说明配置成功了, 如果不行,可以重启下试试。(注意,你的.tex文档应该与zhfontcfg.sty文件同在一个文件夹)。

\end{enumerate}

\end{document}
