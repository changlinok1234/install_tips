\documentclass[a4paper,12pt]{article}
%%--------------------------------------------------------------------
%packages
\usepackage{zhfontcfg}
\linespread{1.5} %行间距
\title{Matlab2012a的安装}
\author{常琳(整理)}
\date{2015-8-21}
\begin{document}
\maketitle


\section{安装}

\begin{enumerate}

\item 挂载iso文件

sudo  mount  -o  loop  路径Mathworks.Matlab.R2012a.UNIX.iso  /mnt  

\item 进入挂载目录

cd  /mnt

\item 安装

sudo  ./install

\item 出现图形化安装界面,破解说明在挂载目录下面的  crack  文件夹,建议安装到/opt/matlab目录下

\item 安装完成后,可以用用命令行打开matlab

sudo  /opt/matlab/bin/matlab

\item 解决The  desktop  configuration  was  not  saved  successfully
问题的关键就在于Permission  Denied,因此解决方法就是给目录和文件加权限,
代码:

sudo  chmod  a+w  -R  ~/.matlab

这一步也可以以后用到的时候写

\end{enumerate}

\section{创建快捷方式}

\begin{enumerate}
\item 将附件里的  matlab.desktop  文件放在  /usr/share/applications  下,图片  matlab.png  放在  /usr/share/icons  下

可是使用sudo  cp,也可以使用命令:

sudo  nautilus

进入文件管理器临时使用root权限,否则在用户账户下面没有权限在除  /home  之外的其他目录进行删除复制等操作。

\item 建立软链接使快捷方式生效

代码:

sudo  ln  -s  /opt/matlab/bin/matlab    /usr/bin/matlab

\end{enumerate}

\section{解决中文乱码问题}

\begin{enumerate}

\item 字体显示美化  进入Matlab,从菜单打开:Files->preferences,打开Fonts页,把右边最下面的复选框Use  antialising  to  smooth  desktop  fonts选中,重启MATLAB,字体显示的效果就很好了.

\item MATLAB使用自带的Java运行环境,根据CPU架构的不同,相对应的字体配置文件路径为:

32位版本  /usr/local/matlab/sys/java/jre/glnx86/jre/lib/fontconfig.properties

64位版本  /usr/local/matlab/sys/java/jre/glnxa64/jre/lib/fontconfig.properties

下面以64位版本为例进行配置

\item 进入字体配置文件目录
代码:
cd  /opt/matlab/sys/java/jre/glnx86/jre/lib

如果fontconfig.properties文件不存在,可以从fontconfig.properties.src复制一个
代码:
sudo  cp  fontconfig.properties.src  fontconfig.properties

效果图如附件里的  1.png

\item 字体可直接用系统自带的文泉驿
修改JRE的字体配置文件,打开配置文件:
代码:
  sudo  gedit  fontconfig.properties

加入中文字体定义,在version=1下面一行输入

allfonts.chinese-arphic1=-misc-simsun-medium-r-normal--0-0-0-0-p-0-iso10646-1

接着指明中文字体路径,在allfonts.chinese-arphic1行后回车另起一行,输入中文字体文件的完整路径:

filename.-misc-simsun-medium-r-normal--0-0-0-0-p-0-iso10646-1=/usr/share/fonts/truetype/wqy/wqy-microhei.ttc

效果图如附件里的  2.png

\item 修改字体搜索,  接着在此配置文件中查找  sequence.allfonts  行。如果其后的sequence开头的行中有:  chinese-arphics1,  可以略过此步;否则在其后面加入一行:  sequence.fallback=chinese-arphic1

效果图如附件里的  3.png

\end{enumerate}

完成。

参考网站:http://forum.ubuntu.org.cn/viewtopic.php?t=373776


\end{document}


