\documentclass[a4paper,12pt]{article}
%%--------------------------------------------------------------------
%packages
\usepackage{zhfontcfg}
\defaultfontfeatures{Mapping=tex-text}  %%如果没有它,会有一些 tex 特殊字符无法正常使用,比如连字符。
\linespread{1.5} %行间距
\title{Codeblocks与Opencv的安装与配置}
\author{常琳(整理)}
\date{2015-8-21}
\begin{document}
\maketitle

\section{codeblocks安装}

\begin{enumerate}

\item 安装基本编译环境

sudo apt-get install build-essential

sudo apt-get install

\item 安装codeblocks

sudo apt-get install codeblocks

sudo apt-get install codeblocks-dbg

sudo apt-get install wxformbuilder

\item 安装wxWidgets

sudo apt-get install libwxbase2.8

sudo apt-get install libwxbase2.8-dev

sudo apt-get install libwxgtk2.8-0

sudo apt-get install libwxgtk2.8-dev

sudo apt-get install libwxgtk2.8-dbg

sudo apt-get install wx-common

sudo apt-get install wx2.8-headers

sudo apt-get install wx2.8-i18n

wx2.8-examples和wx2.8-doc想看文档可以装上

\item 配置codeblocks

打开codeblocks

settings->global variables

current variable标签后面点击new按钮,出来的框框填写wx

然后 built-in fields下面

base /usr

include /usr/include/wx-2.8

lib /usr/lib

点击close保存 

\end{enumerate}

\section{安装opencv}

\begin{enumerate}

\item 系统更新

sudo apt-get update

sudo apt-get upgrade

\item 安装必要的组件

sudo apt-get install build-essential libgtk2.0-dev libjpeg-dev libtiff4-dev libjasper-dev libopenexr-dev cmake python-dev python-numpy python-tk libtbb-dev libeigen3-dev yasm libfaac-dev libopencore-amrnb-dev libopencore-amrwb-dev libtheora-dev libvorbis-dev libxvidcore-dev libx264-dev libqt4-dev libqt4-opengl-dev sphinx-common texlive-latex-extra libv4l-dev libdc1394-22-dev libavcodec-dev libavformat-dev libswscale-dev default-jdk ant libvtk5-qt4-dev

\item 编译opecv3.0

进入解压后的opencv3.0目录

mkdir build

进入build目录

cd build

cmake -D WITH\_TBB=ON -D BUILD\_NEW\_PYTHON\_SUPPORT=ON -D WITH\_V4L=ON -D INSTALL\_C\_EXAMPLES=ON -D INSTALL\_PYTHON\_EXAMPLES=ON -D BUILD\_EXAMPLES=ON -D WITH\_QT=ON -D WITH\_OPENGL=ON ..


\item 安装opencv3.0

在build目录

\begin{enumerate}

\item make  

make编译失败,出现如下错误:

make[2 ]:*** [modules/java/core+Algorithm-jdoc.java] 错误126

make[1]:*** [modules/java/CmakeFiles/opencv\_java.dir/all ]错误2

make:*** [all] 错误 2

搜网上解决方案,把CMakeLists.txt里的有关java的--java support--和==java==部分删掉了,影响未知

\item sudo make install

\end{enumerate}

\item 配置opencv.conf file ,加入环境变量  

sudo gedit /etc/ld.so.conf.d/opencv.conf

在opencv.conf里面加入命令

/usr/local/lib

更新库目录

sudo ldconfig

打开文件bash.bashrc  //这个暂时不知道是干嘛的,我没执行, 程序没影响

sudo gedit /etc/bash.bashrc

加入下面两行

PKG\_CONFIG\_PATH=\$PKG\_CONFIG\_PATH:/usr/local/lib/pkgconfig

export PKG\_CONFIG\_PATH

\end{enumerate}

\section{配置}

\begin{enumerate}
\item codeblocks链接库配置:Project->Build Options

test1(工程名称)->Linker settings

在Link libraries下面写

/../../../usr/lib/libopencv\_core.so

/../../../usr/lib/libopencv\_highgui.so

以上仅是列举,lib文件里的都要加进去

\item codeblocks头文件目录配置 (pkg-config --cflags opencv 结果)

test1->Search directories->Compiler

/usr/include/opencv

/usr/include/opencv2

include里opencv的包都要加上

\item CodeBlocks 路文件目录配置

test1->Search directories->Linker

/usr/lib

安装好后,用testprogram里的.cbp文件测试下

\end{enumerate}

参考网址:http://blog.csdn.net/adong76/article/details/40018407
          
         http://blog.csdn.net/hitwengqi/article/details/7985343

\end{document}

