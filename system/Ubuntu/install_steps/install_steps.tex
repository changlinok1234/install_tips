
\documentclass[a4paper,12pt]{article}
%%--------------------------------------------------------------------
%packages
\usepackage{zhfontcfg}

%---------------------------------------------------------------------
\linespread{1.5} %行间距
\defaultfontfeatures{Mapping=tex-text}  %%如果没有它,会有一些 tex 特殊字符无法正常使用,比如连字符。
\title{Linux14.04安装}
\graphicspath{{figures/}}  

%---------------------------------------------------------------------------
\begin{document}
\maketitle
\begin{enumerate}
\item 进入BIOS,如果用U盘安装的话设置成U盘优先启动,如果用光盘安装设置成光盘优先启动,这里是用U盘安装。完成设置后保存退出。

\item 进入安装界面,语言选择中文简体。选择“安装Ubuntu”进入下一个界面。

\item 安装第三方软件那里可以先不选,点击“继续”到下一界面。

\item 安装类型的选择,初次安装,可以选择“其他选项”,点击“继续”。

\item 之后会出现分区界面,单机左下方的“-”会删除分区,“+”会出现创建分区对话框,实验室一般分/boot(单系统不用)、/swap(内存4G以上不用考虑)、/、/home、/opt(网上查得的先后顺序,不一定百分百正确,目前使用无状况)。/boot 一般分100-200M,/swap为内存的1.5-2倍。以800G硬盘空间为例,在创建分区对话框中:

“挂载点”/boot:250M \qquad 在“用于”一栏选择“Ext4日志文件系统”

/swap:8G \qquad 在“用于”一栏选择“交换空间”

“挂载点”/:100G \qquad 在“用于”一栏选择“Ext4日志文件系统”

“挂载点”/home:400G \qquad 在“用于”一栏选择“Ext4日志文件系统”

“挂载点”/opt:剩下的292G \qquad 在“用于”一栏选择“Ext4日志文件系统”
分区类型我都选的逻辑分区,新分区的位置选择空间起始位置
\item 设置好分区后单击继续,键盘布局选择英语(美国),单击继续。

\item 填写个人信息,单击继续。

\item 系统更新

sudo apt-get update

sudo apt-get upgrade


\end{enumerate}


\end{document}
